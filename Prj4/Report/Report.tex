\documentclass[conference]{IEEEtran}
\IEEEoverridecommandlockouts
% The preceding line is only needed to identify funding in the first footnote. If that is unneeded, please comment it out.
\usepackage[utf8]{inputenc} % allow utf-8 input
\usepackage[T1]{fontenc}    % use 8-bit T1 fonts
\usepackage{hyperref}       % hyperlinks
\usepackage{url}            % simple URL typesetting
\usepackage{booktabs}       % professional-quality tables
\usepackage{nicefrac}       % compact symbols for 1/2, etc.
\usepackage{microtype}      % microtypography
\usepackage{multicol}
\usepackage{multirow}
\usepackage{diagbox}
\usepackage{subfigure}
\usepackage{tabularx}
\usepackage{cite}
\usepackage{amsmath,amssymb,amsfonts}
\usepackage{algorithmic}
\usepackage{graphicx}
\usepackage{textcomp}
\usepackage{xcolor}
\usepackage{indentfirst}
\def\BibTeX{{\rm B\kern-.05em{\sc i\kern-.025em b}\kern-.08em
    T\kern-.1667em\lower.7ex\hbox{E}\kern-.125emX}}
\begin{document}
\bibliographystyle{IEEEtran}

\title{Principles of Data Science Project 4\\
Domain Adaptation}

\author{\IEEEauthorblockN{Hongzhou Liu}
\IEEEauthorblockA{517030910214}
\texttt{deanlhz@sjtu.edu.cn}
\and
\IEEEauthorblockN{Xuanrui Hong}
\IEEEauthorblockA{517030910227}
\texttt{hongxuanrui.1999@sjtu.edu.cn}
\and
\IEEEauthorblockN{Qilin Chen}
\IEEEauthorblockA{517030910155}
\texttt{1017856853@sjtu.edu.cn}
}

\maketitle

\begin{abstract}
In this project, we tried different domain adaptation methods on the Office-Home dataset, which contains 65 categories of things from 4 domains.
The four domains are Art, Clipart, Product and Real-World. In our experiments, we take Art, Clipart and Product as source domains and Real-World as target domain.
For traditional methods, we tried KMM, CORAL, GFK, TCA, EasyTL and BDA. For deep learning methods, we only tried DAN due to the scarce of computation resources and time limitation.
We compared performances among those methods and discussed the difference among them. 
\end{abstract}

\begin{IEEEkeywords}
Domain Adaptation, Transfer Learning
\end{IEEEkeywords}

\section{Introduction}
\cite{lowe1999object}
\bibliography{Prj4}
\end{document}