\documentclass{article}
% if you need to pass options to natbib, use, e.g.:
%     \PassOptionsToPackage{numbers, compress}{natbib}
% before loading neurips_2020

% ready for submission
% \usepackage{neurips_2020}

% to compile a preprint version, e.g., for submission to arXiv, add add the
% [preprint] option:
%     \usepackage[preprint]{neurips_2020}

% to compile a camera-ready version, add the [final] option, e.g.:
%     \usepackage[final]{neurips_2020}

% to avoid loading the natbib package, add option nonatbib:
\usepackage[preprint]{neurips_2020}
\usepackage[utf8]{inputenc} % allow utf-8 input
\usepackage[T1]{fontenc}    % use 8-bit T1 fonts
\usepackage{hyperref}       % hyperlinks
\usepackage{url}            % simple URL typesetting
\usepackage{booktabs}       % professional-quality tables
\usepackage{amsfonts}       % blackboard math symbols
\usepackage{nicefrac}       % compact symbols for 1/2, etc.
\usepackage{microtype}      % microtypography
\usepackage{multicol}
\usepackage{xeCJK}

\title{Principles of Data Science Project 1\\
        Dimension Reduction}

% The \author macro works with any number of authors. There are two commands
% used to separate the names and addresses of multiple authors: \And and \AND.
%
% Using \And between authors leaves it to LaTeX to determine where to break the
% lines. Using \AND forces a line break at that point. So, if LaTeX puts 3 of 4
% authors names on the first line, and the last on the second line, try using
% \AND instead of \And before the third author name.

\author{
  Hongzhou Liu \\
  517030910214 \\
  \texttt{deanlhz@sjtu.edu.cn} \\
  \And
  Xuanrui Hong \\
  517030910227 \\
  \texttt{aaa@bbb.ccc} \\
  \And
  Qilin Chen \\
  517030910155 \\
  \texttt{aaa@bbb.ccc} \\
}

\newcommand{\fix}{\marginpar{FIX}}
\newcommand{\new}{\marginpar{NEW}}

\begin{document}

\maketitle

\textbf{To Xuanrui Hong:}\\
请先完成Method部分\\
截止日期:周六晚24点\\
提交方式:\texttt{git}\\
\begin{itemize}
  \item 可以使用ppt上的图片、公式,使用公式必须要完整体现这个方法的内涵
  \item 可以参考维基百科,但需要做一定的改写, i.e. paragraph
  \item 参考文献获取途径:前往\texttt{sci-kit learn}官方网站,搜索对应的方法,如PCA,在相应文档页面查找是否有参考文献。或者找到左上角User Guide,寻找相应内容,如2.Unsupervised Learning-2.5.Decomposing signals in components (matrix factorization problems)
  \item 可以参考往届报告,行文风格、参考文献等,但不必每个方法处处都引用
\end{itemize}

\begin{abstract}
  TODO: Hongzhou Liu\\
\end{abstract}

\section{Method}
\subsection{Feature Selection}
\subsubsection{Select-k-best}
TODO: Xuanrui Hong\\
\subsubsection{Variance Threshold}
TODO: Xuanrui Hong\\
\subsubsection{Tree-based Selection}
TODO: Xuanrui Hong\\

\subsection{Feature Projection}
\subsubsection{PCA}
TODO: Xuanrui Hong\\
\subsubsection{LDA}
TODO: Xuanrui Hong\\

\subsection{Feature Learning}
\subsubsection{t-SNE}
TODO: Xuanrui Hong\\
\subsubsection{LLE}
TODO: Xuanrui Hong\\
\subsubsection{AutoEncoder}
TODO: Xuanrui Hong\\

\section{Experiment}
\subsection{Baseline}
TODO: Hongzhou Liu\\
\subsection{Feature Selection}
\subsubsection{Select-k-best}
TODO: Xuanrui Hong\\
\subsubsection{Variance Threshold}
TODO: Xuanrui Hong\\
\subsubsection{Tree-based Selection}
TODO: Xuanrui Hong\\

\subsection{Feature Projection}
\subsubsection{PCA}
TODO: Qilin Chen\\
\subsubsection{LDA}
TODO: Qilin Chen\\

\subsection{Feature Learning}
\subsubsection{t-SNE}
TODO: Qilin Chen\\
\subsubsection{LLE}
TODO: Qilin Chen\\
\subsubsection{AutoEncoder}
TODO: Qilin Chen\\

\section{Conclusion}
TODO: Qilin Chen\\
\section*{Acknowledgement}

\end{document}